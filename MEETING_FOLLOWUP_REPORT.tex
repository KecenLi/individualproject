% !TEX root = MEETING_FOLLOWUP_REPORT.tex
% Compile with: xelatex -interaction=nonstopmode MEETING_FOLLOWUP_REPORT.tex
\documentclass[11pt]{article}

\usepackage[a4paper,margin=1in]{geometry}
\usepackage{graphicx}
\usepackage{float}
\usepackage{booktabs}
\usepackage{hyperref}
\usepackage{enumitem}
\usepackage{xcolor}
\usepackage{caption}
\usepackage{subcaption}

% Chinese support (XeLaTeX)
\usepackage{fontspec}
\usepackage{xeCJK}
\setmainfont{DejaVu Serif}
\setmonofont{DejaVu Sans Mono}
\setCJKmainfont{Noto Sans CJK SC}
\setCJKmonofont{Noto Sans CJK SC}

\hypersetup{
  colorlinks=true,
  linkcolor=blue,
  urlcolor=blue
}

\title{Meeting Follow-Up Report (Unified Edition)\\
\large 非技术版详细说明 + 技术版 Summary(统一、详尽)}
\author{Kecen Li}
\date{\today}

\begin{document}
\maketitle
\tableofcontents
\newpage

% ================================================================
\section{一页式总览(先给“听得懂的人”)}
\textbf{一句话目标:}用\textbf{Neuron Activation Coverage (NAC)}评估模型在不同扰动下的内部神经元行为,判断“模型是否在走原来的正确路径”,并验证它与鲁棒性/泛化能力的关系。

\textbf{本次工作完成了什么?}
\begin{itemize}[leftmargin=2em]
  \item 已\textbf{完整跑通官方 NAC-UE 评测流程}(OpenOOD / ood\_coverage),并与非 APS(不使用 OOD 校准)对照。
  \item 已\textbf{完成 OODRobustBench 的 CIFAR-10-C 全量评估}(15 corruption × 5 severity)。
  \item 已\textbf{完成自然分布偏移(CIFAR-10.1 / 10.2)评估}。
  \item 已\textbf{完成 Phase3 组合扰动与顺序敏感性实验}并生成可视化。
  \item 已对核心结论与局限性进行归档,论文素材整理到 \texttt{archive/2026-02-02\_paper\_materials/}。
\end{itemize}

\textbf{一句话结论:}NAC 在\textbf{明显分布偏移/噪声类扰动}上表现较好,但对\textbf{近域自然偏移}和\textbf{部分对抗扰动}的区分度有限;APS 提供上界,但非 APS / ID-only 更接近真实应用场景。

% ================================================================
\section{先把 NAC 讲清楚(结合论文原始定义)}
这部分严格依据 ICLR 2024 论文 \emph{Neuron Activation Coverage: Rethinking Out-of-Distribution Detection and Generalization}。以下解释既保留数学定义,也尽量用直白语言讲清楚。

\subsection{通俗解释:我们到底在看什么?}
传统 OOD 检测看的是\textbf{模型输出的置信度},比如最大 softmax。NAC 不仅看输出,而是看\textbf{模型内部某一层的神经元行为},特别是“这些神经元在做出当前判断时到底有多重要”。

简化理解:
\begin{itemize}[leftmargin=2em]
  \item 如果一个输入让模型内部的神经元激活方式\textbf{和训练数据相似},它就更像“正常输入”。
  \item 如果激活方式\textbf{偏离训练时的分布},就更像“异常输入”。
\end{itemize}

\subsection{论文公式(核心定义)}
论文首先定义一个神经元的“激活状态”(不仅是输出值,还包含其对决策的影响):

\begin{equation}
\hat{z} = \sigma\left(z \odot \frac{\partial D_{KL}(u||p)}{\partial z}\right)
\end{equation}
其中:
\begin{itemize}[leftmargin=2em]
  \item $z$ 是某层神经元的原始输出;
  \item $D_{KL}(u||p)$ 是模型输出 $p$ 与均匀分布 $u$ 的 KL divergence;
  \item $\partial D_{KL}/\partial z$ 衡量该神经元对模型决策的影响;
  \item $\sigma$ 是 sigmoid,$\alpha$ 控制其陡峭程度。
\end{itemize}

接下来定义 NAC(覆盖度函数):
\begin{equation}
\Phi_i^{X}(\hat{z}_i; r)=\frac{1}{r}\min\left(\kappa_i^{X}(\hat{z}_i), r\right)
\end{equation}
\begin{itemize}[leftmargin=2em]
  \item $\kappa_i^{X}(\cdot)$ 表示在训练集 $X$ 上的概率密度(PDF);
  \item $r$ 是覆盖度“饱和阈值”:超过 $r$ 就认为“充分覆盖”;
  \item 直觉:\textbf{训练数据经常出现的神经元状态 → 高覆盖;罕见状态 → 低覆盖}。
\end{itemize}

\subsection{NAC-UE(用于 OOD 检测)}
NAC-UE 就是把所有神经元的覆盖度平均:
\begin{equation}
S(x^*; \hat{f}, X)=\frac{1}{N}\sum_{i=1}^N \Phi_i^{X}(\hat{f}(x^*)_i; r)
\end{equation}
若 $S$ 较低,则说明该输入触发了训练中很少见的神经元状态 → 更可能是 OOD。

\subsection{NAC-ME(用于模型评估/泛化)}
论文还提出 NAC-ME:通过对 NAC 分布积分来衡量模型的“覆盖面积”,从而预测其泛化能力(尤其是 OOD 泛化)。

\paragraph{论文重要细节:}
\begin{itemize}[leftmargin=2em]
  \item 参数:$\alpha$(sigmoid 陡峭度),$r$(覆盖饱和阈值),$M$(PDF 分桶数)。
  \item 参数敏感性:论文指出 $r$ 很敏感,$\alpha$ 越陡通常越好。
  \item 模型选择:ResNet 通常用 layer4,ViT 用 block-11 attention 作为 NAC 层。
  \item 训练数据:论文强调使用\textbf{训练数据(正确分类样本)}建立 NAC 统计分布。
\end{itemize}

% ================================================================
\section{本项目如何对齐论文(从概念到实现)}
\subsection{项目 pipeline(通俗解释 + 技术映射)}
\textbf{通俗解释:}我们把流程拆成 3 步:
\begin{enumerate}[leftmargin=2em]
  \item \textbf{扰动器}:把图片变“坏”,包括噪声、压缩、几何变形、对抗攻击。
  \item \textbf{分析器}:让模型前向 + 反向,算出神经元覆盖度。
  \item \textbf{对比分析}:对比 clean vs. 各类扰动,看 NAC 是否能区分它们。
\end{enumerate}

\textbf{技术映射:}
\begin{itemize}[leftmargin=2em]
  \item NAC 计算:\texttt{src/nac\_efficient.py} + \texttt{src/official\_nac.py}。
  \item 攻击/扰动:AutoAttack、Advex-UAR、DeepG、OODRobustBench。
  \item 模型:RobustBench 的标准/鲁棒模型。
  \item 实验脚本:\texttt{scripts/run\_phase3\_correct.py}, \texttt{run\_total\_benchmark.py}, \texttt{scripts/run\_oodrb\_nac.py}。
\end{itemize}

\subsection{与论文一致的关键点(已落实)}
\begin{itemize}[leftmargin=2em]
  \item 使用论文定义的 $\hat{z}$(含 KL 梯度)和 NAC 函数。
  \item 使用训练集构造覆盖度分布(profiling),测试集做评估。
  \item 多层 NAC 评估(ResNet block3 / layer4),并可扩展到 ViT。
  \item APS(Automatic Parameter Search)与非 APS 两条流程都跑通。
\end{itemize}

\subsection{数据集、模型与评估设置(必须明确的“材料清单”)}
为避免误读,以下列出本报告中\textbf{实际使用}的数据集与设置:
\textbf{ID 数据:}
\begin{itemize}[leftmargin=2em]
  \item CIFAR-10(训练集用于 NAC profiling;测试集用于评估与比较)。
\end{itemize}
\textbf{OOD 数据(OpenOOD / ood\_coverage):}
\begin{itemize}[leftmargin=2em]
  \item Near-OOD:CIFAR-100、TinyImageNet (TIN)。
  \item Far-OOD:MNIST、SVHN、Texture、Places365。
\end{itemize}
\textbf{Corruption 数据(OODRobustBench / CIFAR-10-C):}
\begin{itemize}[leftmargin=2em]
  \item 15 种腐蚀 × 5 个严重度,共 75 条结果(完整表格见 CSV)。
\end{itemize}
\textbf{自然分布偏移(OODRobustBench Natural Shift):}
\begin{itemize}[leftmargin=2em]
  \item CIFAR-10.1、CIFAR-10.2。
\end{itemize}
\textbf{模型与评估设置:}
\begin{itemize}[leftmargin=2em]
  \item 模型:RobustBench 标准 ResNet18 与鲁棒 WideResNet (Gowal2021)。
  \item Phase3 设置:profiling=1000,test=2000,batch=64,目标层=block3,AA 版本=standard。
  \item OODRB 设置:batch=64,profiling=1000,ID 测试=10000。
  \item 指标:AUROC、FPR@95、AUPR\_IN、AUPR\_OUT、Accuracy、NAC-Acc 相关性。
\end{itemize}

% ================================================================
\section{实验结果(结合图表,逐块讲清楚)}

\subsection{A. 官方 OOD Coverage(APS vs Non-APS)}
\textbf{通俗解释:}APS 像“拿到了部分 OOD 的答案再调参数”,因此更强;非 APS 更接近现实。我们必须报告两者差异。

\textbf{关键结果(Non-APS):}
\begin{itemize}[leftmargin=2em]
  \item Near-OOD(CIFAR-100/TIN):AUROC \textbf{89.83–92.02},FPR@95 \textbf{26.56–35.10}。
  \item Far-OOD(MNIST/SVHN/Texture/Places365):AUROC \textbf{91.85–96.05},FPR@95 \textbf{14.34–26.73}。
  \item 汇总指标:Near-OOD 平均 AUROC ≈ \textbf{90.93};Far-OOD 平均 AUROC ≈ \textbf{94.60}。
\end{itemize}

\begin{figure}[H]
  \centering
  \includegraphics[width=0.95\textwidth]{ood_coverage/analysis/nac_aps_vs_nonaps.png}
  \caption{APS vs Non‑APS 结果对比(CIFAR-10, NAC-UE)}
\end{figure}

\subsection{B. Top/Bottom 质检(定性解释)}
\textbf{通俗解释:}我们不是只看数字,还看“分数最高/最低的样本长什么样”。如果低分样本确实“奇怪”,说明 NAC 是可信的。

\begin{figure}[H]
  \centering
  \begin{subfigure}{0.48\textwidth}
    \centering
    \includegraphics[width=\textwidth]{ood_coverage/analysis/nac_top_bottom_20260127_130431/top100_grid.png}
    \caption{ID Top‑100(高 NAC)}
  \end{subfigure}
  \hfill
  \begin{subfigure}{0.48\textwidth}
    \centering
    \includegraphics[width=\textwidth]{ood_coverage/analysis/nac_top_bottom_20260127_130431/bottom100_grid.png}
    \caption{ID Bottom‑100(低 NAC)}
  \end{subfigure}
  \caption{ID 样本的 Top/Bottom 质检}
\end{figure}

\begin{figure}[H]
  \centering
  \begin{subfigure}{0.48\textwidth}
    \centering
    \includegraphics[width=\textwidth]{ood_coverage/analysis/nac_ood_top_bottom_nonaps_mnist_20260127_134547/top100_grid.png}
    \caption{OOD Top‑100(MNIST, Non‑APS)}
  \end{subfigure}
  \hfill
  \begin{subfigure}{0.48\textwidth}
    \centering
    \includegraphics[width=\textwidth]{ood_coverage/analysis/nac_ood_top_bottom_nonaps_mnist_20260127_134547/bottom100_grid.png}
    \caption{OOD Bottom‑100(MNIST, Non‑APS)}
  \end{subfigure}
  \caption{OOD 样本质检(MNIST)}
\end{figure}

\subsection{C. ID-only 阈值(真实场景的最低保障)}
\textbf{通俗解释:}真实应用中我们无法用 OOD 样本调参,所以我们只用 ID 数据设阈值,看 OOD 能否被检测出来。
\textbf{关键数值(来自 threshold\_table.csv):}
\begin{itemize}[leftmargin=2em]
  \item 1\% ID-FPR 阈值:Near-OOD (CIFAR-100/TIN) TPR \textbf{0.118–0.152}。
  \item 5\% ID-FPR 阈值:Near-OOD TPR \textbf{0.452–0.514}。
  \item Far-OOD(SVHN/Texture)在 5\% 阈值时 TPR 通常 \textbf{0.66–0.78}。
\end{itemize}

\begin{figure}[H]
  \centering
  \includegraphics[width=0.95\textwidth]{ood_coverage/analysis/nac_idonly_thresholds_20260127_140612/threshold_table.png}
  \caption{ID-only 阈值(1\% / 5\%)下的 OOD 检出率}
\end{figure}

\subsection{D. OODRobustBench:CIFAR-10-C 全量结果}
\textbf{结果总结:}
\begin{itemize}[leftmargin=2em]
  \item 完整评估 75 条(15 corruption × 5 severity)。
  \item 平均 AUROC = \textbf{0.698},平均 FPR@95 = \textbf{0.779}(见 oodrb\_nac\_20260202\_054121.csv)。
  \item 最容易被检出的:强噪声 / 对比度 / 结构破坏类(如 impulse\_noise, contrast)。
  \item 最难被检出的:亮度 / 雾化(低强度接近 0.5,接近随机)。
\end{itemize}

\subsection{E. 自然分布偏移(CIFAR-10.1 / 10.2)}
\textbf{结果:}AUROC 平均 \textbf{0.563},FPR@95 平均 \textbf{0.930}(见 oodrb\_nac\_20260202\_205623.csv)。

\textbf{结论(通俗解释):}自然偏移是“很像但不完全一样”的分布变化,NAC 在这种情况下几乎失效,接近随机水平。

\subsection{F. Phase3 组合扰动与顺序效应}
\textbf{通俗解释:}我们不仅看单一扰动,还看“先转再加噪声”和“先加噪声再转”是否不同。

\begin{figure}[H]
  \centering
  \includegraphics[width=0.95\textwidth]{archive/2026-02-02_paper_materials/phase3_output_20260202/histogram_block3.png}
  \caption{Phase3 NAC 分布直方图(block3)}
\end{figure}

\begin{figure}[H]
  \centering
  \includegraphics[width=0.95\textwidth]{archive/2026-02-02_paper_materials/phase3_output_20260202/comparison_bars.png}
  \caption{Phase3 对比条形图}
\end{figure}

\begin{figure}[H]
  \centering
  \includegraphics[width=0.95\textwidth]{archive/2026-02-02_paper_materials/phase3_output_20260202/delta_heatmap.png}
  \caption{Phase3 变化幅度热力图}
\end{figure}

\textbf{关键观察(数值化):}
\begin{itemize}[leftmargin=2em]
  \item NAC-Acc 相关系数 = \textbf{0.6894}(排除 clean)。
  \item 最大 NAC 降幅来自 AutoAttack 与 JPEG(如 AutoAttack Linf, JPEG L1/L2)。
  \item DeepG 几何扰动 NAC 变化接近 0(部分为正)。
  \item 组合顺序差异非常小(A→B 与 B→A 差异 \textbf{约 1e-3})。
\end{itemize}

\subsection{G. 标准 vs 鲁棒模型对照(归档结果)}
\textbf{现象:}
\begin{itemize}[leftmargin=2em]
  \item 鲁棒模型在 AutoAttack 下准确率明显更高。
  \item 在 Rotation / Fog / Snow 等扰动下,鲁棒模型的 NAC 分离更明显。
\end{itemize}

% ================================================================
\section{讨论:这些结果说明了什么?}
\begin{itemize}[leftmargin=2em]
  \item NAC 对\textbf{明显分布偏移}有效(噪声/结构破坏),对\textbf{近域自然偏移}弱。
  \item APS 可以显著提高表面成绩,但具有过拟合风险(官方 README 明确提醒)。
  \item 非 APS / ID-only 更接近现实场景,是必须报告的“真实能力”。
  \item 组合扰动与顺序效应在 NAC 上影响不大,说明 NAC 更多体现“幅度型变化”而非“路径顺序”。
\end{itemize}

% ================================================================
\section{局限性与风险(必须提前说明)}
\begin{itemize}[leftmargin=2em]
  \item 论文官方 README 有明确“overfitting issue”警告,APS 结果应视作上界。
  \item NAC 对自然偏移(CIFAR-10.1/10.2)检测能力弱,接近随机。
  \item 当前 profiling 默认使用训练集,不一定仅保留“正确分类样本”(与论文设定略有差别)。
\end{itemize}

% ================================================================
\section{可复现材料路径(给导师或审稿人查看)}
\begin{itemize}[leftmargin=2em]
  \item 统一素材目录:\texttt{archive/2026-02-02\_paper\_materials/}
  \item Phase3 输出:\texttt{phase3\_output\_20260202/}
  \item OODRB Corruption CSV:\texttt{oodrb\_results\_20260202/oodrb\_nac\_20260202\_054121.csv}
  \item Natural Shift CSV:\texttt{oodrb\_results\_20260202/oodrb\_nac\_20260202\_205623.csv}
  \item APS/Non‑APS 分析:\texttt{ood\_coverage/analysis/}
\end{itemize}

% ================================================================
\section{后续工作(还可以继续做的)}
\begin{itemize}[leftmargin=2em]
  \item JPEG 扫描结果合并,生成完整曲线(已拆 job)。
  \item ResNet50 / ViT-B/16 层名确认后扩展实验。
  \item ImageNet 版本 OODRB(需要 ImageNet 数据)。
\end{itemize}

\end{document}
