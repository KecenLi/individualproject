\documentclass[a4paper,11pt]{article}
\usepackage[utf8]{inputenc}
\usepackage{geometry}
\usepackage{graphicx}
\usepackage{booktabs}
\usepackage{float}
\usepackage{hyperref}
\usepackage{amsmath}
\usepackage{xeCJK}
\setCJKmainfont{Noto Sans CJK SC}
\setCJKsansfont{Noto Sans CJK SC}
\setCJKmonofont{Noto Sans CJK SC}

\geometry{left=2.5cm,right=2.5cm,top=2.5cm,bottom=2.5cm}

\title{\textbf{Neural Activation Coverage (NAC) Follow-Up 实验报告}}
\author{项目组 - 汇报总结}
\date{\today}

\begin{document}

\maketitle

\section{执行摘要 (Executive Summary)}

本次工作旨在全面验证 Neural Activation Coverage (NAC) 在分布外检测 (OOD Detection) 与模型鲁棒性评估中的实际效能。针对导师提出的核心疑虑,我们完成了官方流程复现、APS/Non-APS 对照、ID-only 阈值验证及大规模 OODRobustBench 扫测。

\textbf{关键结论如下:}
\begin{enumerate}
    \item \textbf{验证了 NAC 的现实可用性}:明确区分了 APS(上界)与 Non-APS(现实)性能。虽然去除了 OOD 验证集校准(APS)后,Near-OOD 检测能力有所下降,但在 Far-OOD 上仍保持约 94\% AUROC。
    \item \textbf{界定了 NAC 的适用边界}:通过 CIFAR-10-C 全量扫测发现,NAC 对\textbf{结构破坏与纹理噪声}(如 Gaussian Noise, Impulse Noise)极其敏感(AUROC 可达 0.90+),但对\textbf{全局光照变化}(如 Brightness, Fog)几乎不敏感(AUROC $\approx$ 0.50)。
    \item \textbf{揭示了自然漂移的挑战}:在 Natural Shift (CIFAR-10.1/10.2) 实验中,NAC 平均 AUROC 约为 0.563,表明仅凭神经元覆盖度难以区分近域的自然分布漂移。
    \item \textbf{排除了顺序效应干扰}:Phase 3 实验证实,扰动叠加顺序(Order A vs B)对 NAC 覆盖度影响微乎其微($\Delta \approx 10^{-3}$),证明该指标对最终状态具备稳定性。
\end{enumerate}

\section{背景与方法定义 (Background \& Methodology)}

\subsection{NAC 核心定义}
NAC (Neural Activation Coverage) 并非仅关注模型输出 logits,而是通过量化\textbf{神经元激活状态的分布覆盖率}来衡量模型对输入的熟悉程度。

基于 ICLR 2024 (NAC) 论文及代码实现 (KMNC),我们采用如下定义:
对任意神经元 $k$,将其激活值范围划分为 $M$ 个区间(Bins)。在训练过程中,记录每个区间是否被至少 $O$ 个样本激活。由此得到每个神经元的覆盖分布 $P_k$。

对于测试样本 $x$,其 NAC-UE (Uncertainty Estimation) 得分定义为“激活状态的加权覆盖度”:
\begin{equation}
    S_{NAC}(x) = \frac{1}{N} \sum_{k=1}^{N} C_{k}(\text{bin}(a_k(x)))
\end{equation}
其中,$a_k(x)$ 是样本 $x$ 在神经元 $k$ 上的激活值,$\text{bin}(\cdot)$ 映射其落入的区间,$C_k(\cdot)$ 是该区间在训练集上的归一化覆盖率。
\begin{itemize}
    \item \textbf{得分高}:表示该样本激发了训练中常见的神经元状态(In-Distribution)。
    \item \textbf{得分低}:表示该样本激发了罕见或未见过的状态(OOD)。
\end{itemize}

\subsection{参数设置}
实验采用 ResNet-18 (CIFAR-10 预训练),提取 \texttt{layer4} (倒数第二层) 特征。关键超参数:
\begin{itemize}
    \item \textbf{M (Bins)}: 1000 (细粒度划分激活空间)
    \item \textbf{Method}: NAC-UE (用于 OOD 检测), NAC-ME (用于泛化评估)
\end{itemize}

\section{实验结果与分析 (Experimental Results)}

\subsection{1. APS vs Non-APS:校准的代价}
为了回应“是否过拟合”的质疑,我们对比了 Automatic Parameter Search (APS) 与 Non-APS 模式。
\begin{itemize}
    \item \textbf{APS (上界)}:使用 OOD 验证集优化超参,Near-OOD 结果出现过拟合现象(TPR 99.0\% 占位值),不可作为现实参考。
    \item \textbf{Non-APS (现实)}:完全不接触 OOD 数据。结果显示 Far-OOD 性能依然稳健,但 Near-OOD 分离度下降。
\end{itemize}

\begin{figure}[H]
    \centering
    \includegraphics[width=0.9\linewidth]{ood_coverage/analysis/nac_aps_vs_nonaps.png}
    \caption{APS 与 Non-APS 模式下的 OOD 检测性能对比。可见 Non-APS 更真实地反映了方法在未知环境下的表现。}
    \label{fig:aps_vs_nonaps}
\end{figure}

\textbf{结论}:报告中后续所有结论均基于 \textbf{Non-APS} 设定,以保证结论的诚实性。

\subsection{2. ID-only 阈值验证}
即便不使用 OOD 数据调参,我们是否能仅通过 ID (In-Distribution) 验证集确定检测阈值?
我们在 ID 验证集上设定 FPR (False Positive Rate) 为 1\% 和 5\% 的阈值,测试 OOD 检出率 (TPR)。

\begin{figure}[H]
    \centering
    \includegraphics[width=1.0\linewidth]{ood_coverage/analysis/nac_idonly_thresholds_20260127_140612/threshold_table.png}
    \caption{仅基于 ID 数据设定的阈值在不同 OOD 数据集上的检出率 (TPR)。}
    \label{fig:id_thresholds}
\end{figure}

数据表明:
\begin{itemize}
    \item 对于 \textbf{Far-OOD} (SVHN, Texture),使用 5\% ID-FPR 阈值可获得约 \textbf{70-77\%} 的检出率,具备实用价值。
    \item 对于 \textbf{Near-OOD} (CIFAR-100),检出率降至 \textbf{11-15\%},说明仅凭 ID 分布很难界定近域边界。
\end{itemize}

\subsection{3. 鲁棒性全景扫测 (OODRobustBench)}
我们运行了 CIFAR-10-C 全量测试 (15 Corruptions $\times$ 5 Severities),揭示了 NAC 的响应特性。

\textbf{发现一:对结构破坏敏感}
Impulse Noise, Gaussian Noise, Pixelate 等破坏图像局部结构的扰动,随强度增加 (Severity 1 $\to$ 5),NAC AUROC 单调上升 (最高达 0.93+)。

\textbf{发现二:对全局光照钝感}
Brightness, Fog 等全局变换,即使强度很高,NAC AUROC 仍停留在 0.50-0.60 之间(接近随机猜测)。

此结果解释了 NAC 的工作原理:它依赖于\textbf{特征提取器的模式匹配}。如果扰动仅改变像素统计 (光照) 而不破坏卷积特征的激活模式,NAC便无法检测。

\subsection{4. 定性分析:NAC 到底“看”到了什么?}
通过对 ID 和 OOD 数据进行 Top/Bottom 采样可视化验证方法的可解释性。

\begin{figure}[H]
    \centering
    \includegraphics[width=0.8\linewidth]{ood_coverage/analysis/nac_top_bottom_20260127_130431/top100_grid.png}
    \caption{ID 数据集中 NAC 得分最高 (Top) 与最低 (Bottom) 的样本示例。}
    \label{fig:top_bottom}
\end{figure}

\begin{itemize}
    \item \textbf{Top Score (High Confidence)}: 主要是背景清晰、物体完整的典型样本 (Canonical examples)。
    \item \textbf{Bottom Score (Low Confidence)}: 包含复杂背景、特殊角度或遮挡的样本,且这些样本往往更容易被模型误分类。
\end{itemize}
这证明 NAC 分数忠实地反映了“模型对当前样本的特征熟悉程度”。

\subsection{5. Phase 3: 组合扰动与顺序效应}
针对多重扰动场景,我们测试了旋转、噪声及其组合顺序的影响。

\begin{figure}[H]
    \centering
    \includegraphics[width=0.9\linewidth]{archive/2026-02-02_paper_materials/phase3_output_20260202/comparison_bars.png}
    \caption{Phase 3 不同扰动组合对 NAC 覆盖度的影响。注意 AutoAttack 与 JPEG 造成的显著下降。}
    \label{fig:phase3_bars}
\end{figure}

\begin{itemize}
    \item \textbf{顺序无关性}:Order A (Rotate $\to$ Noise) 与 Order B (Noise $\to$ Rotate) 的 NAC 差异仅在 $10^{-3}$ 量级,说明 NAC 对最终扰动状态鲁棒,而不受生成路径影响。
    \item \textbf{对抗攻击显著性}:AutoAttack ($L_\infty$) 导致 NAC 均值从 0.95 骤降至 0.76,证明对抗样本不仅改变 logits,也剧烈破坏了内部特征分布。
\end{itemize}

\section{局限性与讨论 (Limitations)}

\textbf{1. APS 的过拟合风险}
论文原版使用的 APS 策略在小规模 OOD 验证集上极其容易过拟合。我们的实验表明,一旦移除 OOD 验证集,NAC 在 Near-OOD 任务上的表现会回落到普通水平。这提示在实际部署时不能盲目信任 APS 指标。

\textbf{2. 自然漂移 (Natural Shift) 的检测盲区}
针对 CIFAR-10.1 / 10.2 的测试显示 AUROC $\approx$ 0.56。这是 NAC 的主要短板:当分布漂移非常微妙(属于语义一致但采集来源不同的近域漂移)时,内部神经元激活并未发生显著改变,导致检测失效。

\section{结论 (Conclusion)}
本阶段工作成功建立了 NAC 的完整评测基准。我们证实了 NAC 是一种基于“特征熟悉度”的有效检测指标,尤其擅长识别破坏图像结构的强 OOD 样本及对抗攻击。然而,其对全局光养变化及近域自然漂移的检测能力有限。后续建议结合输入空间的统计量(如像素直方图)以弥补其在光照鲁棒性上的不足。

\end{document}
