\documentclass[a4paper,11pt]{article}
\usepackage[utf8]{inputenc}
\usepackage{geometry}
\usepackage{graphicx}
\usepackage{booktabs}
\usepackage{float}
\usepackage{hyperref}
\usepackage{amsmath}
\usepackage{amssymb}
\usepackage{xcolor}
\usepackage{xeCJK}

% 字体设置
\setCJKmainfont{Noto Sans CJK SC}
\setCJKsansfont{Noto Sans CJK SC}
\setCJKmonofont{Noto Sans CJK SC}

\geometry{left=2.5cm,right=2.5cm,top=2.5cm,bottom=2.5cm}

\title{\textbf{个人项目进度报告: Week 2 总结}}
\author{Kecen Li}
\date{\today}

\begin{document}

\maketitle

\begin{abstract}
本报告全面总结了 Week 2 的工作进展,涵盖了 Week 1 中完成的关键工程基础设施建设,以及基于此得出的最新实验分析见解。我们详细阐述了如何构建一个统一的评估基础设施,以解决主流学术库(RobustBench, Advex-UAR, OpenOOD)之间的数据协议冲突。在此稳健的平台之上,我们要对神经元激活覆盖(NAC)指标进行了深度刻画,揭示了其在不同扰动类型下的“敏感度光谱”,以及在鲁棒模型上的反直觉行为。此外,我们澄清了自动参数搜索(APS)与非 APS 模式之间的关键区别,并确立了更符合现实场景的评估基准。
\end{abstract}

\tableofcontents
\newpage

\section{简介与工程回顾 (Introduction \& Retrospective)}

在上次讨论后,我意识到 Week 1 所建立的工程基础设施的复杂性未能得到充分传达。在展示新数据之前,我想简要概述一下确保所有后续结果有效性的“隐藏”工程层。

为了进行科学严谨的评估,我集成了三个本不兼容的独立学术库:
\begin{enumerate}
    \item \textbf{RobustBench}: 用于获取标准化的、排行榜级别的模型架构(标准 ResNet vs 鲁棒 WideResNet)。
    \item \textbf{Advex-UAR / AutoAttack}: 提供最先进的对抗攻击和常见腐蚀算法。
    \item \textbf{OpenOOD (NAC)}: 用于计算神经元激活覆盖度指标的核心库。
\end{enumerate}

由于输入协议的冲突,直接集成是不可行的。例如,\texttt{Advex-UAR} 强制要求输入经过 ImageNet 统计数据(均值/方差)的归一化,而 \texttt{RobustBench} 模型则期望原始的 $[0,1]$ 张量。为解决此问题,我实现了一个\textbf{透明中间件层 (Transparent Middleware Layer)}(位于 \texttt{src/loader.py} 和 \texttt{src/perturber.py}),它能自动处理可逆的归一化转换。这确保了当我们应用“$L_\infty$ 攻击”时,攻击是在正确的像素空间内数学精确地执行的,从而避免无效的实验伪影。

此外,我还开发了一个\textbf{架构兼容垫片 (Architecture Compatibility Shim)} (\texttt{src/official\_nac.py})。原始 NAC 代码针对特定 ResNet 层进行了硬编码。我的兼容层允许我们动态挂钩任意架构——包括视觉 Transformer (ViT)——而无需修改核心库。这项能力目前已经“准备就绪”,可用于未来的对比研究。

\section{方法论: 神经元激活覆盖 (NAC)}

\subsection{定义与机制}
神经元激活覆盖 (NAC) 是本研究的核心指标。与依赖最终输出概率 (logits) 的传统 OOD 检测器不同,NAC 衡量的是\textbf{无限层特征激活的“熟悉度”}。

\begin{itemize}
    \item \textbf{概念}: 对于 $L$ 层中的给定神经元 $k$,我们将训练期间观察到的激活范围离散化为 $M$ 个区间 (Bins)。如果可靠的训练样本频繁激活某个区间,则该区间被视为“已覆盖”。
    \item \textbf{推理}: 对于测试样本 $x$,验证通过检查其诱导的激活是否落入这些高覆盖率区间来执行。
    \item \textbf{公式}: NAC 分数 $S(x)$ 是所有受监控神经元的平均覆盖概率:
    \begin{equation}
        S_{NAC}(x) = \frac{1}{N} \sum_{k=1}^{N} C_{k}(\text{bin}(a_k(x)))
    \end{equation}
    其中 $a_k(x)$ 是神经元 $k$ 的激活值,$C_k(\cdot)$ 是源自训练数据的覆盖频率。
\end{itemize}

\subsection{评估模式: APS vs. Non-APS}
我们评估中的一个关键区别是参数调整策略。ICLR 原论文提出了\textbf{自动参数搜索 (APS)}。
\begin{itemize}
    \item \textbf{APS (“上界”)}: 使用一个小的 OOD 验证集来调整超参数(如阈值 $\alpha$)。虽然这能产生很高的性能指标(例如在基准测试中报告近乎完美的分离度),但我们的分析表明它存在\textbf{过拟合}验证集中特定 OOD 类型的风险。
    \item \textbf{Non-APS (“现实基准”)}: 我们严格避免使用 OOD 数据进行调参,仅依赖分布内 (ID) 验证数据。这反映了未来异常性质未知的真实部署场景。
\end{itemize}
\textbf{决策}: 除非另有说明,本报告中的所有结果现在默认使用 \textbf{Non-APS} 或仅 ID 设置,以确保报告鲁棒性时的诚实性。

\section{实验设置 (Experimental Setup)}

为确保可重复性与透明度,我们在此明确定义本项目使用的数据集与核心参数。

\subsection{数据集清单 (Datasets)}
本研究使用了多组数据集来探测检测器的边界:
\begin{itemize}
    \item \textbf{ID (In-Distribution)}: CIFAR-10。
    \item \textbf{Near-OOD (近域分布)}: CIFAR-100、TinyImageNet (TIN)。它们与 ID 语义相似但类别不同。
    \item \textbf{Far-OOD (远域分布)}: MNIST、SVHN、Texture、Places365。它们的统计特征与 ID 截然不同。
    \item \textbf{Corruption (腐蚀)}: CIFAR-10-C (OODRobustBench),包含 15 种腐蚀类型 $\times$ 5 种强度(共 75 种变体)。
    \item \textbf{Natural Shift (自然漂移)}: CIFAR-10.1、CIFAR-10.2。真实采集的自然分布漂移数据集。
    \item \textbf{Phase 3}: CIFAR-10 的特定子集,用于测试受控的“几何 + 对抗”组合扰动。
\end{itemize}

\subsection{实施细节 (Implementation Details)}
\begin{itemize}
    \item \textbf{样本量}: 
    \begin{itemize}
        \item Profiling(训练集): 1,000 样本。
        \item Evaluation(测试集): 10,000 样本(全量测试),或 2,000 样本(Phase 3 快速扫测)。
    \end{itemize}
    \item \textbf{Batch Size}: 128。
    \item \textbf{目标层}: ResNet-18 的 \texttt{layer4}(最后一个卷积 Block)。
    \item \textbf{模式区分}: 
    \begin{itemize}
        \item \textbf{APS}: 仅作为“理论上界”参考。
        \item \textbf{Non-APS}: 作为主要的“现实基准”,不使用 OOD 数据调参。
    \end{itemize}
\end{itemize}

\section{实验结果与分析 (Experimental Results \& Analysis)}

\subsection{关键量化结果 (Key Statistics)}
在深入分析前,先列出核心量化指标:
\begin{itemize}
    \item \textbf{OODRobustBench (CIFAR-10-C)}: 
    \begin{itemize}
        \item 平均 AUROC $\approx \mathbf{0.698}$
        \item 平均 FPR@95 $\approx \mathbf{0.779}$
    \end{itemize}
    \item \textbf{Natural Shift (CIFAR-10.1/10.2)}:
    \begin{itemize}
        \item 平均 AUROC $\approx \mathbf{0.563}$
        \item 平均 FPR@95 $\approx \mathbf{0.93}$
    \end{itemize}
    \item \textbf{APS vs. Non-APS 差异}:
    \begin{itemize}
        \item Near-OOD: APS 可达 $\approx \mathbf{99\%}$ AUROC; Non-APS 降至 $\approx \mathbf{89-92\%}$。
        \item Far-OOD: 两种模式均保持高水准 ($\approx \mathbf{92-96\%}$)。
    \end{itemize}
    \item \textbf{Phase 3 相关性}: NAC 分数与模型准确率 (Accuracy) 的相关系数为 $\mathbf{0.689}$。
\end{itemize}

\subsection{敏感度光谱: NAC 到底检测到了什么?}
我们使用 \textbf{OODRobustBench (CIFAR-10-C)} 套件进行了全面扫测。结果揭示了 NAC 清晰的“敏感度光谱”。

\begin{figure}[H]
    \centering
    \includegraphics[width=0.95\linewidth]{archive/2026-02-02_paper_materials/phase3_output_20260202/comparison_bars.png}
    \caption{不同扰动类型对 NAC 覆盖度的影响。注意 AutoAttack/JPEG 引起的显著下降。}
    \label{fig:sensitivity}
\end{figure}

\subsubsection{高敏感区 (结构/纹理破坏)}
NAC 在检测破坏局部空间结构或高频纹理信息的扰动方面极其有效。
\begin{itemize}
    \item \textbf{扰动类型}: 脉冲噪声 (Impulse Noise)、像素化 (Pixelate)、$L_\infty$ 对抗攻击。
    \item \textbf{表现}: 随着严重程度增加 (1 $\rightarrow$ 5),NAC AUROC 分数单调增加,通常达到 $>0.90$。
    \item \textbf{原因}: 这些腐蚀从根本上改变了卷积神经网络 (CNN) 的局部感受野,导致激活落入“未见过”的区间。
\end{itemize}

\subsubsection{低敏感区 (全局/低频漂移)}
相反,NAC 在处理保留局部结构的全局变换时非常吃力。
\begin{itemize}
    \item \textbf{扰动类型}: 亮度 (Brightness)、雾 (Fog)、雪 (Snow)。
    \item \textbf{表现}: 即使在最大严重程度下,AUROC 仍保持在 $0.50$ 附近(相当于随机猜测)。
    \item \textbf{原因}: 全局亮度偏移本质上是一种线性变换。CNN(特别是带有 BatchNorm 的)被设计为对此类偏移具有不变性。因此,内部激活的变化不足以触发 NAC 检测器。
\end{itemize}

\subsection{Phase 3: 组合与顺序效应}
为了回答关于组合扰动的问题(“顺序 A $\to$ B 是否不同于 B $\to$ A?”),我们运行了特定的组合实验(例如,旋转 + 噪声)。
\begin{itemize}
    \item \textbf{发现}: 顺序 A 和顺序 B 之间的 NAC 覆盖差异可以忽略不计($\Delta \approx 10^{-3}$)。
    \item \textbf{结论}: NAC 对生成的\textit{历史路径}具有鲁棒性;它只关心特征的最终状态。
\end{itemize}

\subsection{鲁棒性悖论 (The Robustness Paradox)}
也许最反直觉的发现是对比\textbf{标准 ResNet18} 与\textbf{对抗训练 WideResNet (Gowal2021)}。

\begin{table}[H]
\centering
\begin{tabular}{lcc}
\toprule
\textbf{场景} & \textbf{标准模型 (AUROC)} & \textbf{鲁棒模型 (AUROC)} \\
\midrule
高斯噪声 (Gaussian Noise) & \textbf{0.90} (被检测到) & 0.65 (难以检测) \\
AutoAttack ($L_\infty$) & \textbf{0.76} (被检测到) & 0.52 (未被检测) \\
\bottomrule
\end{tabular}
\caption{标准模型与鲁棒模型上的 NAC 检测性能对比。}
\label{tab:paradox}
\end{table}

\textbf{洞察}: 分类器的“鲁棒性”意味着其内部特征对扰动具有不变性。然而,NAC 依赖\textit{变化量}(特征破坏)来检测异常。因此,\textbf{一个对抗鲁棒的模型通过稳定其特征,有效地向 NAC 检测器“隐藏”了攻击}。这提出了一个根本性的权衡:更好的分类器(更鲁棒)可能拥有一个效果更差的 OOD 检测器(基于激活的方法)。

\section{自然漂移的局限性 (Natural Shift Analysis)}
作为边界测试,我们评估了 \textbf{自然漂移} (CIFAR-10.1 / CIFAR-10.2)。
\begin{itemize}
    \item \textbf{结果}: AUROC $\approx \mathbf{0.563}$, FPR@95 $\approx \mathbf{0.93}$。
    \item \textbf{结论}: NAC 无法区分这些自然漂移与原始训练数据。这证实了 NAC 检测的是“异常性”而非微妙的“分布漂移”。
\end{itemize}


\section{需求完成度自检 (Progress Checklist)}

我们将当前的进展与基于 Xiyue 邮件要求的项目路线图进行对照:

\begin{table}[H]
\centering
\begin{tabular}{p{0.3\linewidth} p{0.53\linewidth} p{0.1\linewidth}}
\toprule
\textbf{需求任务} & \textbf{实施状态} & \textbf{完成度} \\
\midrule
指标: NAC (OpenOOD) & 已集成官方 v1.5 并配备自定义架构垫片。 & \textcolor{teal}{\textbf{已完成}} \\
\midrule
对抗与腐蚀扰动 & AutoAttack ($L_\infty, L_2$) + Advex-UAR (Elastic, Fog, Snow, Gabor, JPEG) 全面扫测。 & \textcolor{teal}{\textbf{已完成}} \\
\midrule
OOD Corruption & OODRobustBench (CIFAR-10-C) 全量扫测(75 种变体)已完成。 & \textcolor{teal}{\textbf{已完成}} \\
\midrule
分析范围 & 纯净 / 单一 / 组合 / 顺序效应 (Phase 3) 分析完成。 & \textcolor{teal}{\textbf{已完成}} \\
\midrule
几何变换 (DeepG) & 基础设施已编译 (libgeometric.so);旋转/平移已测试。高级形变待大范围扫测。 & \textcolor{orange}{\textbf{部分完成}} \\
\midrule
基准模型 (RobustBench) & CIFAR-10 (ResNet18, WRN-28-10) 已完成。ImageNet/ViT 因资源/权限待定。 & \textcolor{orange}{\textbf{部分完成}} \\
\bottomrule
\end{tabular}
\caption{实际交付组件与导师要求的对照表。}
\label{tab:checklist_cn}
\end{table}

\section{结论与未来展望 (Conclusion \& Future Capabilities)}
我们已经成功建立了一个严谨、数学一致的基准测试流水线。
\begin{enumerate}
    \item \textbf{工程层面}: 中间件、DeepG 集成和架构垫片已全面运行。
    \item \textbf{科学层面}: 通过大量的 CIFAR-10-C 扫测,我们将 NAC 定性为“结构异常检测器”,并量化了其边界。
    \item \textbf{下一步}: 当前的基础设施已准备好,一旦数据和配置就绪,即可启动 ViT 与集成 NAC 的实验。
\end{enumerate}

\end{document}
